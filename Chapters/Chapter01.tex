\chapter{はじめに}\label{ch:intro}



論文の組版において \LaTeX やWord\mynote{Wordは苦手です。}と格闘しておられる方も多いのではないでしょうか。
このテンプレートは、
\texttt{classicthesis} パッケージ  4.8 (\url{https://ctan.org/pkg/classicthesis?lang=en}) をベースに、我々の執筆ニーズに合わせてカスタマイズしたものです。

主な変更点は以下の通りです:
\begin{itemize}
    \item APA 第7版の引用スタイル
    \item 日本語著者名のカスタムフォーマット
    \item 日本語文献のソート機能
    \item 多言語サポート
\end{itemize}


\autorefja{ch:examples}と\autorefja{ch:cat}では、実例を用いて様々な \LaTeX の機能を紹介します。

\section{クイックスタート}

\textbf{すぐに執筆を始めたいですか?} 基本的な手順は以下の通りです:

\begin{enumerate}
    \item \textsf{情報の入力}: \texttt{config.tex} (14行目以降) を編集し、名前、タイトル、所属などを入力します。

    \item \textsf{執筆}: \texttt{Chapters/Chapter01.tex}、\texttt{Chapter02.tex} などに章ごとのコンテンツを記述します。

    \item \textsf{文献の構築}: \texttt{Bibliography.bib} (または \texttt{part1.bib}、\texttt{part2.bib} など) に参考文献エントリを追加します。

    \item \textsf{画像の追加}: 図表を \texttt{gfx/} フォルダに配置します。

    \item \textsf{コンパイル}: メニュー (左上) を開き、コンパイラを LuaLaTeX に設定します。

    \item \textsf{Configuration Options}: In  texttt{config.tex}, you can enable specific settings. Enable \texttt{drafting} to print the version information on the first page of the thesis. Or set the metadata.
    
    \item \textsf{Version History}: In \texttt{VersionHistory.tex}, change the version history details.
\end{enumerate}



\section{英文論文の場合}

英文がメイン場合は下記のテンプレートを使用してください。

\begin{center}
\url{https://github.com/0ldriku/ClassicThesis_EN}
    
\end{center}
